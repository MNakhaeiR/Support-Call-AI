\documentclass{article}
\usepackage[utf8]{inputenc}
\usepackage{graphicx}
\usepackage{hyperref}

\title{Project Documentation}
\author{[Your Name]}
\date{\today}

\begin{document}

\maketitle

\section{Overview}
This document provides an overview of the project, including its purpose, features, and architecture.

\section{Installation}
To install the project, follow these steps:
\begin{enumerate}
    \item Clone the repository from GitHub.
    \item Navigate to the project directory.
    \item Install the required dependencies using the command:
    \begin{verbatim}
    pip install -r requirements.txt
    \end{verbatim}
\end{enumerate}

\section{Usage}
To run the application, execute the following command:
\begin{verbatim}
python src/main.py
\end{verbatim}
This will start the main application interface.

\section{Features}
\begin{itemize}
    \item Audio capture and processing
    \item Sentiment analysis
    \item Profanity detection
    \item Emotion analysis
    \item Stress detection
    \item Integration with large language models
\end{itemize}

\section{Architecture}
The project is structured into several modules:
\begin{itemize}
    \item \textbf{src/}: Contains the main application code.
    \item \textbf{config/}: Holds configuration files.
    \item \textbf{models/}: Stores machine learning models.
    \item \textbf{data/}: Contains data files for recordings, logs, and analysis results.
    \item \textbf{tests/}: Includes unit tests for the application.
    \item \textbf{docs/}: Contains documentation files.
\end{itemize}

\section{Contributing}
Contributions are welcome! Please submit a pull request or open an issue for any enhancements or bug fixes.

\section{License}
This project is licensed under the MIT License. See the LICENSE file for more details.

\end{document}